\documentclass[a4paper]{IEEEtran}

\usepackage{xcolor}
\usepackage{hyperref}
\usepackage[utf8]{inputenc}

\newcommand\TODO[1]{\textcolor{red}{TODO:#1}}
\newcommand\todo[1]{\TODO{#1}}
\newcommand\cn{\textcolor{red}{[citation needed]}}

\title{The overpowered prefetcher of goodness and excellence!}

\author{
    Sigve Sebastian Farstad,
    Rune Holmgren,
    Torbjørn Langland,
    Per Thomas Lundal
}

\begin{document}

\maketitle

\begin{abstract}
    Le abstract.\cn
\end{abstract}

\section{Introduction}

\todo{ Write briefly about the project, the course it belongs to and the boal}
\break
This is the introduction section. Here we write something appealing for you to continue on. If you finish this, you will be baked a cake.
\break
\break
As part of the course TDT4260 Computer Architecture, students in groups of 3 or 4 were to implement and test a prefetcher. This report describes the implementation with test results done by group 11. The report will describe the implemented prefetcher, how it was implemented, the framework and the test results.

\section{Related Work}
\todo{ Not sure, mention related work?}
\break
Here we mention related work. Do you want cake for your relatives?
\section{Prefetcher Description}
\todo{ Describe how the final prefetcher works. I suggest adding a figure. Maybe briefly mention other attempteps while if we have space?}
\break
Here we describe the cake...uh, I mean the pretecher. Hey, Vance! Can you prefetch that cake?
\section{Methodology}
\todo{ Mention the framework. Explain PFJudge. Maybe or maybe not mention C++?}
\break

Here we describe how we did it.

WARNING: Recipee too long. Insert cake recipe here!
\section{Results}
\todo{ Describe results from both local and PFJudge.}
\break

EXPLOSION!
\section{Discussion}
\todo{ Not exaclty sure, just say it works better? Compare with other prefetcher IF we chose to describe them. }
\break


You killed GLaDOS.
You monster.
\section{Conclusion}
\todo{ Mention what could have been done better or different? Future ideas? }
\break

The cake is a lie!
\section{Acknowledgements (optional)}
GLaDOS

yoyo \cite{assignment-text}

\bibliography{bibliography}
\bibliographystyle{plain}
\nocite{*}

\end{document}
